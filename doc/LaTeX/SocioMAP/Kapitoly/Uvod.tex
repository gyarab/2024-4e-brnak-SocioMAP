\chapter{Úvod}
Předmětem tohoto ročníkového projektu bylo vytvořit aplikaci v Android Studiu v jazyce Java, která slouží jako sociální síť pro vyhledávání všemožných událostí v blízkém okolí.

Nejpřirozenější způsob, jak člověk může objevit události kolem sebe, je pomocí mapy. Proto je aplikace postavena především na interaktivní mapě. Během vývoje jsem se rozhodl přidat i možnost interakce mezi uživateli, přestože to původní zadání nepožadovalo. Uvědomil jsem si totiž, že sociální síť bez vzájemné komunikace není tak úplně sociální sítí.

Aplikace je navržena tak, aby byla přívětivá všem. Zpočátku byla aplikace mířena na dorostence a dospívající, jelikož osoby v tomto věku často tráví svůj volný čas venku společně se svými vrstevníky. Informace o událostech získávají ze sociálních sítí jako je Instagram, ale aplikace není přímo navržena pro vyhledávání akcí, nebo vyhledávají akce na internetu.

Aplikaci lze dále rozšiřovat o mnoho dalších funkcí. Většina z nich je již implementována, ale před zveřejněním v Google Play by bylo vhodné doladit zbývající chybějící prvky a vyladit funkčnost.
