\chapter{Instalace}

Tato kapitola popisuje, jakým způsobem lze aplikaci SocioMapa připravit a spustit v prostředí vývojáře. Jsou zde uvedeny všechny potřebné nástroje, nastavení a konfigurace.

\section{Android a Git}

Pro vývoj a správu aplikace SocioMapa je potřeba mít nainstalované následující nástroje, jimiž jsou Android Studio – vývojové prostředí pro Android aplikace, Java Development Kit (JDK) – doporučuje se verze 17 nebo novější, Git – pro klonování repozitáře a správu verzí.


\section{Firebase (backend)}

Aplikace využívá Firebase jako backend pro autentizaci uživatelů, ukládání dat (Firestore) a další služby.

\subsection*{Postup nastavení Firebase:}
\begin{enumerate}
	\item Vytvořte projekt na https://console.firebase.google.com.
	\item Přidejte Android aplikaci do projektu pomocí balíčkového jména aplikace.
	\item Stáhněte konfigurační soubor google-services.json a vložte ho do složky app/ v Android projektu.
	\item Ujistěte se, že v souboru build.gradle (Project) i build.gradle (Module: app) jsou správně přidány závislosti Firebase a plugin google-services.
	\item V konzoli Firebase nastavte authentication (Email/Password + Google), Firestore databázi (v režimu testování nebo s upravenými pravidly)
	\item Upravte pravidla přístupu do Firestore podle požadavků na zabezpečení.
\end{enumerate}



\section{Postup instalace}
\begin{enumerate}
    \item Klonujte repozitář aplikace.
    \item V Android Studiu otevřete složku s klonovaným projektem.
    \item Nechte Android Studio automaticky provést synchronizaci závislostí (Gradle Sync).
    \item Nastavte Firebase viz. \textit{Postup nastaveni Firebase:}
    \item Vytvořte API pro GoogleMaps a změňte API key v aplikaci 
\end{enumerate}

